\textbf{1.} Considera la siguiente gramática:
\[S \rightarrow aSbS | bSaS | \epsilon\]
Construye dos derivaciones por la derecha para la cadena abaabb.\newline
¿Cómo es el árbol se sintaxis concreta para esta cadena, es único?
\newline

\textbf{Solución.} A continuación se dan las dos derivaciones por la derecha requeridas, estas son
\begin{multicols}{2}
        Derivación 1.
        \begin{eqnarray*}
                S &\rightarrow& aSbS \\
                &\rightarrow& abSaSb\epsilon\\
                &\rightarrow& ab\epsilon aaSbSb\\
                &\rightarrow& abaa\epsilon b\epsilon b\\
                &\rightarrow& abaabb
        \end{eqnarray*}

        Derivación 2.
        \begin{eqnarray*}
                S &\rightarrow& aSbS \\
                &\rightarrow& a\epsilon baSbS\\
                &\rightarrow& abaaSbSb\epsilon\\
                &\rightarrow& abaa\epsilon b\epsilon b\\
                &\rightarrow& abaabb
        \end{eqnarray*}
\end{multicols}
Los árboles de sintáxis abstracta se muestran a continuación
\begin{multicols}{2}
Derivación 1.
    \begin{center}
      \begin{forest}
        [S [a] [S [b] [S [$\epsilon$]] [a] [S [a] [S [$\epsilon$]] [b] [S [$\epsilon$]]]] [b] [S [$\epsilon$]]]
      \end{forest}
    \end{center}

Derivación 2.
    \begin{center}
      \begin{forest}
        [S [a] [S [$\epsilon$]] [b] [S [a] [S [a] [S [$\epsilon$]] [b] [S [$\epsilon$]]] [b] [S [$\epsilon$]]]]
      \end{forest}
    \end{center}
\end{multicols}

Como se puede ver, el árbol de sintáxis abstracta no es único.
