\documentclass{article}

% Symbols
\usepackage{amsfonts, amsthm}
\usepackage{upgreek}
\usepackage{physics}
\usepackage{cancel}
\usepackage{amssymb, latexsym, amsmath}
\usepackage{ stmaryrd }

%Algorithms
\usepackage[ruled,lined,linesnumbered,commentsnumbered]{algorithm2e}

%% Identación
\setlength{\parindent}{0cm}

% Tree
\usepackage{tree-dvips}
\usepackage{qtree}
\usepackage[linguistics]{forest}

% Comentario en bloques:
\iffalse
\fi

% Hipervínculos:
\usepackage{hyperref}

% Código
\newcommand{\code}[1]{\textcolor{white!25!black}{\texttt{#1}}}
\usepackage{listings}

%AMS
\usepackage{amsthm}
\newtheorem{algo-thm}{Algoritmo}

% Proof
\renewcommand*{\proofname}{\textbf{Soluci\'on:}}
% Theorem
\newtheorem*{theorem}{Teorema}

% Graphics
\usepackage{graphicx}
\usepackage{pgf}

% Color a letras.
%\usepackage[usenames,dvipsnames,svgnames,table]{xcolor}

% Tikz
\usepackage{tkz-graph}
\usetikzlibrary{arrows,automata}
\usepackage{tikz}
\usetikzlibrary{arrows,automata}
%\usetikzlibrary[topaths]

% Def. Dr. César.
\usetikzlibrary{shapes,calc}
\tikzstyle{edge}=[shorten <=2pt, shorten >=2pt, >=stealth, line width=1.1pt]
\tikzstyle{blueE}=[shorten <=2pt, shorten >=2pt, >=stealth, line width=1.5pt, blue]

\tikzstyle{blackV}=[circle, fill=black, minimum size=6pt, inner sep=0pt, outer sep=0pt]
\tikzstyle{blueV}=[circle, fill=blue, draw, minimum size=6pt, line width=0.75pt, inner sep=0pt, outer sep=0pt]
\tikzstyle{redV}=[circle, fill=red, draw, minimum size=6pt, line width=0.75pt, inner sep=0pt, outer sep=0pt]
\tikzstyle{redSV}=[semicircle, fill=red, minimum size=3pt, inner sep=0pt, outer sep=0pt, rotate=225]
\tikzstyle{blueSV}=[semicircle, fill=blue, minimum size=3pt, inner sep=0pt, outer sep=0pt, rotate=225]
\tikzstyle{blackSV}=[semicircle, fill=black, minimum size=3pt, inner sep=0pt, outer sep=0pt, rotate=225]
\tikzstyle{vertex}=[circle, draw, minimum size=6pt, line width=0.75pt, inner sep=0pt, outer sep=0pt]

% Margins
\addtolength{\voffset}{-1.5cm}
\addtolength{\hoffset}{-1.5cm}
\addtolength{\textwidth}{3cm}
\addtolength{\textheight}{3cm}

% Columnas multiples
\usepackage{multicol}

%Header-Footer
\usepackage{fancyhdr}
\renewcommand{\headrulewidth}{1pt}

\newcommand{\set}[1]{
  \left\{ #1 \right\}
}
\newcommand{\ffst}{\textsc{First}}
\newcommand{\ffollow}{\textsc{Follow}}

%\pagenumbering{gobble} -- Este comando
%                       -- quita el número de página.
\footskip = 50pt
\renewcommand{\headrulewidth}{1pt}

\pagestyle{fancyplain}

\begin{document}
\title{UNIVERSIDAD NACIONAL AUT\'ONOMA DE M\'EXICO\\ Facultad de Ciencias}
\author{Integrantes: \\
  Adri\'an Aguilera Moreno\\
  Sebastián Alejandro Gutierrez Medina}
\date{}
\maketitle
\begin{center}
  \includegraphics[scale=0.20]{../Imagen/Portada}\\[0.4cm]
  \Large
  \textbf{\normalsize}{Compiladores}

\end{center}
\newpage
\fancyhead[r]{ Compiladores 2023-2}
%%%%%%%%%%%%%%%%%%%%%%%%%%%%%%%%%%%%%%%%%%%%%%%%%%%%%
\section*{\LARGE{Tarea 02}}
% Pregunta 1:

\textbf{(2pts.)} Considera la siguiente gram\'atica donde $E$ es el s\'imbolo
inicial:
\[
\begin{array}{rcl}
  E & \to & Aa\\
  A & \to & BC \mid BCf\\
  B & \to & b\\
  C & \to & c\\
\end{array}
\]
\begin{enumerate}
\item Construye los conjuntos can\'onicos de items \textbf{LR(0)}.
\item Con estos conjuntos construye el aut\'omtata finito mostrando las transiciones
  con la funci\'on GoTo.
\item Muestra la tabla de parsing que se genera usando el aut\'omata anterior. 
\end{enumerate}

\newline

%%% Pregunta 02:
Considera el siguiente c\'odigo:
\begin{verbatim}
int mul(int x, int y) {
    if (x)
        return y - (0 - mul(x + 1, y));
   else
        return 0;
}
\end{verbatim}
Proporciona una traducci\'on a RTL suponiendo que ya se han seleccionado 
las instrucciones. \\Deber\'as mostrar las traducciones intermedias de 
subexpresiones del c\'odigo.

\[
    \begin{array}{l l}

        mul(2) & \\
          &  entry: \ L24\\
          &  locals: \ \#10 \ \#11\\
         & \\
        L24 & alloc\_frame \ \rightarrow \ L23\\
        L23 & mov \ \%rbx \ \#10 \ \rightarrow \ L22\\
        L22 & mov \ \%r12 \ \#11 \ \rightarrow \ L21\\
        L21 & mov \ \%rdi \ \#1 \ \rightarrow \ L20\\
        L20 & mov \ \%rdi \ \#2 \ \rightarrow \ L15\\
         & \\
        L15 & mov \ \#1 \ \#8 \ \rightarrow \ L14\\
        L14 & mov \ \#2 \ \#9 \ \rightarrow \ L13\\
        L13 & jle \ \#8 \ \rightarrow \ L12, \ L02\\
         & \\
        L12 & mov \ \#1 \ \#6 \ \rightarrow \ L11\\
        L11 & mov \ \#2 \ \#7 \ \rightarrow \ L10\\
        L10 & add \ \$1 \ \#6 \ \rightarrow \ L09\\
        L09 & goto \ \rightarrow \ L19\\
        L19 & mov \ \#6 \ \%rdi \ \rightarrow \ L18\\
        L18 & mov \ \#7 \ \%rdi \ \rightarrow \ L17\\
        L17 & call \ mul(2) \ \rightarrow \ L16\\
        L16 & mov \ \%rax \ \#4 \ \rightarrow \ L20\\
        L08 & imul \ \$-1 \ \#4 \ \rightarrow \ L07\\
        L07 & add \ \$0 \ \#4 \ \rightarrow \ L06\\
        L06 & mov \ \#2 \ \#5 \ \rightarrow \ L05\\
        L05 & mov \ \#4 \ \#3 \ \rightarrow \ L04\\
        L04 & imul \ \$-1 \ \#3 \ \rightarrow \ L03\\
        L03 & add \ \#5 \ \#3 \ \rightarrow \ L01\\
         & \\
        L01 & goto \ \rightarrow \ L29\\
        L29 & mov \ \#3 \ \%rax \ \rightarrow \ L28\\
        L28 & mov \ \#10 \ \%rbx \ \rightarrow \ L27\\
        L27 & mov \ \#11 \ \%r12 \ \rightarrow \ L26\\
        L26 & delete\_frame \ \rightarrow \ L25\\
        L25 & return \\
        \\
        L02 & mov \ \$0 \ \#3 \ \rightarrow \ L01\\

    \end{array}
\]


\newpage
\textbf{3.} \textbf{(2.5pts.)} Extender la siguiente gram\'atica con atributos para 
la regla $E \to E_1 * E_2$ y obtener el c\'odigo de tres direcciones para la 
expresi\'on $x = a[i] * b[j] $ donde $a$ y $b$ son arreglos de tama\~no 
$2\times 3$ y $2\times 2$ respectivamente y cada uno de ellos almacena enteros 
cuyo tama\~no es $4$.
\begin{center}
\includegraphics[width=.85\textwidth]{./ArrayRef}
\end{center}

\newpage
% Pregunta 4:

\textbf{(Hasta 1.5pts. extra)} Realiza una tabla comparativa entre los estilos
de parsing LL, LR0, SLR y LR1. Incluye caracter\'isticas o descripciones breves
de las coincidencias o diferencias entre ellos. \newline

No olvides agregar las referencias, libros o p\'aginas web, consultadas.

\newpage
\textbf{5.}Considera la siguiente gram\'atica:
\[
    \begin{array}{rcl}
        E & \to & -E \mid (E) \mid VE'\\
        E' & \to & -E \mid \varepsilon\\
        V & \to & \mathtt{id}V'\\
        V' & \to & (E) \mid \varepsilon\\
    \end{array}
\]
\begin{enumerate}

    \item Muestra el c\'alculo de los conjuntos {\ffst} y {\ffollow}.

        \begin{itemize}

            \item FIRST

            \begin{itemize}

                \item FIRST(E) = \{-, (, FIRST(V)\}

                \item FIRST(E') = \{-\}

                \item FIRST(V) = \{i\}

                \item FIRST(V') = \{(\}

            \end{itemize}

            \item FOLLOW

            \begin{itemize}

                \item FOLLOW(E) = \{FIRST(E), FIRST(E')\}

                \item FOLLOW(E') = \{FIRST(E)\}

                \item FOLLOW(V) = \{d\}

                \item FOLLOW(V') = \{FIRST(E)\}

            \end{itemize}

        \end{itemize}

    \item Muestra dos \'arboles de sintaxis, uno abstracta y otro
    concreta para la cadena
    $-\mathtt{id}(-\mathtt{id})-\mathtt{id}$.

        \begin{figure}[h]
            \centering
            \includegraphics[scale = 0.5]{../Imagen/abst}
            \caption{Arbol de Sintaxis Abstracta}
            \label{fig:abst}
        \end{figure}


\end{enumerate}
\end{document}
