\textbf{4.} La siguiente gram\'atica describe un lenguaje de consultas simple en
donde \verb=STRING= es un s\'imbolo terminal:
\[
    \begin{array}{rcl}
        \mathtt{Session} &  \to & \mathtt{Fact}\;\;\mathtt{Session} \\
        \mathtt{Session} &  \to & \mathtt{Question} \\
        \mathtt{Session} &  \to & \mathtt{(\,Session\,)}\;\;\mathtt{Session} \\
        \mathtt{Fact}  & \to & \mathtt{!\,STRING}\\
        \mathtt{Question}  & \to & \mathtt{?\,STRING}
    \end{array}
\]
\begin{enumerate}
    \item[] Calcula las funciones $\ffst$ y $\ffollow$ para los s\'imbolos
    no-terminales de la gram\'atica.
\end{enumerate}

\begin{itemize}

    \item FIRST

        \begin{itemize}

              \item FIRST(Session) = \{FIRST(Fact), FIRST(Question), (\}

              \item FIRST(Fact) = \{!\}

              \item FIRST(Question) = \{?\}

        \end{itemize}

    \item FOLLOW

        \begin{itemize}

            \item FOLLOW(Session) = \{FIRST(Session), STRING\}

            \item FOLLOW(Fact) = \{STRING\}

            \item FOLLOW(Question) = \{STRING\}

        \end{itemize}

\end{itemize}


