\documentclass[11pt,letterpaper]{article}
\usepackage{../packagesComp}
\usepackage{../optionsComp}

\input{../macrosComp}

\title{Compiladores 2023-2\\ Facultad de Ciencias UNAM \\ Tarea 4}
\author{Lourdes Gonz\'alez Huesca\\ Juan Alfonso Gardu\~no Sol\'is \and  
Braulio Aaron Santiago Carrillo  \\Ma. Fernanda Mendoza Castillo}
\date{Entrega: \textbf{jueves 20 abril 2023}~\footnote{Entrega en la 
plataforma classroom del grupo, en equipos de dos o tres personas.}}


\begin{document}

\maketitle

\begin{enumerate}
\item \textbf{(2pts.)} Considera la siguiente gram\'atica donde $E$ es el s\'imbolo
inicial:
\[
\begin{array}{rcl}
E & \to & Aa\\
A & \to & BC \mid BCf\\
B & \to & b\\
C & \to & c\\
\end{array}
\]
\begin{enumerate}
\item Construye los conjuntos can\'onicos de items \textbf{LR(0)}.
\item Con estos conjuntos construye el aut\'omtata finito mostrando las transiciones
con la funci\'on {\fgoto}.
\item Muestra la tabla de parsing que se genera usando el aut\'omata anterior. 
\end{enumerate}

\item Considera la siguiente gram\'atica donde $A$ es el s\'imbolo inicial:
\[
\begin{array}{rcl}
A & \to & bB \\
B & \to & cC \\
B & \to & cCe \\
C & \to & dA\\
A & \to & a\\
\end{array}
\]
\begin{enumerate}
\item \textbf{(1pt.)} Describe formalmente el lenguaje que acepta la gram\'atica.
\item \textbf{(3pts.)} Proporciona el aut\'omata para construir la tabla de 
parsing \textbf{LR(1)}.
\item \textbf{(1pt.)} De ser posible, analiza una cadena no trivial y de longitud 
al menos 4, mostrando la secuencia de acciones del aut\'omata mediante una tabla 
que incluya al menos la actualizaci\'on de la cadena de entrada y la actualizaci\'on
de la pila.
\end{enumerate}


\item \textbf{(3pts.)} Muestra que la siguiente gram\'atica pertenece a la 
clase \textbf{LL(1)} pero no a la clase \textbf{SLR} ($E$ es el s\'imbolo inicial).
\[
E \to A \, a\, A\,b \;\mid\; B\,b\,B\,a \qquad \qquad  A \to \varepsilon 
\qquad\qquad B\to \varepsilon
\]

\item \textbf{(Hasta 1.5pts. extra)} Realiza una tabla comparativa entre los estilos de parsing LL, LR0, SLR y LR1. Incluye caracter\'isticas o descripciones breves de las coincidencias o diferencias entre ellos.  
No olvides agregar las referencias, libros o p\'aginas web, consultadas.



\end{enumerate}

\end{document}
