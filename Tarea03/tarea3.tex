\documentclass[11pt,letterpaper]{article}
\usepackage{../packagesComp}
\usepackage{../optionsComp}

\input{../macrosComp}

\title{Compiladores 2023-2\\ Facultad de Ciencias UNAM \\ Tarea 3}
\author{Lourdes Gonz\'alez Huesca\\ Juan Alfonso Gardu\~no Sol\'is \and  
Braulio Aaron Santiago Carrillo  \\Ma. Fernanda Mendoza Castillo}
\date{Entrega: \textbf{viernes 24 marzo 2023}~\footnote{Entrega en la 
plataforma classroom del grupo, en equipos de dos o tres personas.\newline
Entrega excepcional en viernes debido al d\'ia feriado 20 de marzo.}}


\begin{document}

\maketitle

% \vspace*{-30pt}

\begin{enumerate}
\item Considera la siguiente gram\'atica:
\[
\begin{array}{rcl}
E & \to & -E \mid (E) \mid VE'\\
E' & \to & -E \mid \varepsilon\\
V & \to & \mathtt{id}V'\\
V' & \to & (E) \mid \varepsilon\\
\end{array}
\]
\begin{enumerate}
\item Construye la tabla de parsing para un parser tipo 
LL(1) usando el c\'alculo de los conjuntos {\ffst} y {\ffollow} que obtuviste 
en la tarea anterior. 
\item Muestra 
lo que se obtiene al ejecutar el algoritmo para procesar la cadena 
$-\mathtt{id}(-\mathtt{id})-\mathtt{id}$.
Incluye una tabla para ver el progreso del algoritmo donde se muestre el avance 
del procesamiento de la cadena y la evoluci\'on de la pila del parser. 
\end{enumerate}



\item Considera el siguiente fragmento de una gram\'atica que abstrae el 
comportamiento de expresiones del lenguaje \textsf{C}:

\[
E \to *E \;\mid\; \&E \;\mid\; E = E \;\mid\; E\ -\!> E \;\mid\; id
\]

Esta gram\'atica es ambigua pero se puede transformar en una no-ambigua usando
la precedencia de operadores.
En particular, el acceso a campos de una estructura, $E\ -\!> E$, tiene mayor 
precedencia que la derrefenciaci\'on y las expresiones para direcciones; 
adem\'as, estas tres tienen mayor precedencia que las 
asignaciones~\footnote{Puedes revisar la tabla disponible en esta 
\href{https://justdocodings.blogspot.com/2018/06/operator-precedence-and-associativity.html}{p\'agina} 
para consultar la precedencia y asociatividad de los operadores en \textsc{C}.}
\begin{enumerate}
\item Escribe una gram\'atica equivalente tipo \textbf{LL(1)} que incluya la
precedencia descrita, muestra el proceso o describe las técnicas que uses para 
obtener esta nueva gramática.
\item Muestra la tabla de parsing para la gram\'atica del inciso anterior.
\item Procesa la expresi\'on
\verb!* * a -> b -> c = & * d!  usando el algoritmo para \textbf{LL} y 
mostrando los estados del parser. 
\end{enumerate}

\newpage


\item  \textbf{Hasta 1pt extra} 
Decide  si la siguiente gram\'atica pertenece a la clase LL mediante 
un an\'alisis de la forma de sus producciones. Justifica tu respuesta.\\
 Si no pertenece, ?`es posible transformarla en una de esa clase? Describe la 
idea para hacerlo. 
\[
\begin{array}{rcl}
PROGRAM & \to & \texttt{begin}\; DECLIST\; \texttt{comma} \;STATELIST \;
\texttt{end}\\
DECLIST & \to & d\; \texttt{semi}\; DECLIST \\
DECLIST & \to & d \\
STATELIST & \to & s\; \texttt{semi}\; STATELIST \\
STATELIST &\to &s
\end{array}
\]

\item \textbf{Hasta 1pt extra} 
Muestra que la siguiente gram\'atica pertenece a la clase \textbf{LL(1)}.
\[
S \to A \, a\, A\,b \;\mid\; B\,b\,B\,a \qquad \qquad  A \to \varepsilon 
\qquad\qquad B\to \varepsilon
\]

\end{enumerate}


\end{document}
