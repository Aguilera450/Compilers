\textbf{2.} Considera el siguiente fragmento de una gram\'atica que abstrae el 
comportamiento de expresiones del lenguaje \textsf{C}:

\[
E \to *E \;\mid\; \&E \;\mid\; E = E \;\mid\; E\ -\!> E \;\mid\; id
\]

Esta gram\'atica es ambigua pero se puede transformar en una no-ambigua usando
la precedencia de operadores.
En particular, el acceso a campos de una estructura, $E\ -\!> E$, tiene mayor 
precedencia que la derrefenciaci\'on y las expresiones para direcciones; 
adem\'as, estas tres tienen mayor precedencia que las 
asignaciones~\footnote{Puedes revisar la tabla disponible en esta 
\href{https://justdocodings.blogspot.com/2018/06/operator-precedence-and-associativity.html}{p\'agina} 
para consultar la precedencia y asociatividad de los operadores en \textsc{C}.}
\begin{enumerate}
\item Escribe una gram\'atica equivalente tipo \textbf{LL(1)} que incluya la
precedencia descrita, muestra el proceso o describe las técnicas que uses para 
obtener esta nueva gramática.

\textbf{Solución:} A continuación se muestra la gramática con precedencia (solo un reacomodo)
y con ambig\"uedad:
\[ E \to E\ -\!> E \;\mid\; *E \;\mid\; \&E \;\mid\; E = E \;\mid\; id\]
quitando ambig\"uedades de derecha a izquierda tenemos que
\begin{eqnarray*}
         E &\to& E\ -\!> P\; |\; P\\
         P &\to& *P \;\mid\; \&P \;\mid\; P = P \;\mid\; id
\end{eqnarray*}
ahora, quitemos las ambig\"uedades de izquierda a derecha, esto es
\begin{eqnarray*}
         E &\to& E\ -\!> P\;\mid\; P\\
         P &\to& *A \;\mid\; \&A \;\mid\; A = P \;\mid\; A \\
         %P' &\to& *P' \;\mid\ A\\
         %P'' &\to& \&P'' \;\mid\ A\\
         A &\to& id
\end{eqnarray*}
lo anterio con base a \href{https://barcelonageeks.com/eliminacion-de-ambiguedad-conversion-de-una-gramatica-ambigua-en-gramatica-inequivoca/}{Eliminaci\'on de ambig\"uedad}.

Quitando la recursividad por la izquierda tenemos que
\begin{eqnarray*}
         (0)\;\; E &\to& PE' \;\mid\; P\\
         (1)\; E' &\to& \ -\!> P \;\mid\; \ -\!> E\\
         (2)\; P &\to& *P \;\mid\; \&P \;\mid\; A = P \;\mid\; A \\
         %(3)\; P' &\to& *P' \;\mid\ A\\
         %(4)\; P'' &\to& \&P'' \;\mid\ A\\
         (5)\; A &\to& id
\end{eqnarray*}
La gramática anterior es no recursiva por la izquierda y no ambigua.
\item Muestra la tabla de parsing para la gram\'atica del inciso anterior.

\textbf{Solución:} Ahora construyamos la tabla parsing, esto es
\begin{table}[h]
        \centering
        \caption{FIRST y FOLLOW}
        \label{tab:parsing}
        \begin{tabular}{|c|c|c|}
                \hline
                  & FIRST & FOLLOW \\
                \hline
                 E & $*$, $\&$, id & $=$ \\
                \hline
                 E' & $-\!>$, $*$, $\&$, id & $=$\\
                \hline
                 P & $*$, $\&$, id & $=$\\
                \hline
                 %P' & $*$, id & $=$\\
                 %\hline
                 %P'' & $\&$, id & $=$\\
                 %\hline
                 A & id & $=$\\
                \hline
        \end{tabular}
\end{table}

luego, calculemos la tabla de parsing, a continuación se muestra
\begin{table}[h]
        \centering
        \caption{Tabla de Parsing}
        \label{tab:parsing}
        \begin{tabular}{|c|c|c|c|c|c|}
                \hline
                   & $-\!>$ & $*$ & $\&$ & $=$ & id \\
                \hline
                E  & & (0) & (0) & & (0) \\
                \hline
                E' & (1) & (1) & (1) & & (1) \\
                \hline
                P  & & (2) & (2) & & (2) \\
                %\hline
                %P'  & & (3) & & & (3) \\
                %\hline
                %P''  & & & (4) & & (4) \\
                \hline
                A  & (5) & & & & \\
                \hline
        \end{tabular}
\end{table}
Así, finalizamos este inciso.
\item Procesa la expresi\'on
\verb!* * a -> b -> c = & * d!  usando el algoritmo para \textbf{LL} y 
mostrando los estados del parser.

\textbf{Solución:}
\begin{eqnarray*}
        E &\to& PE'\\
        &\to& *P'E'\\
        &\to& **P'E'\\
        &\to& **idE'\\
        &\to& **aE'\\
        &\to& ** a -\!> E\\
        &\to& ** a -\!> PE'\\
        &\to& ** a -\!> AE'\\
        &\to& ** a -\!> idE'\\
        &\to& ** a -\!> bE'\\
        &\to& ** a -\!> b -\!> E\\
        &\to& ** a -\!> b -\!> P\\
        &\to& ** a -\!> b -\!> A=P\\
        &\to& ** a -\!> b -\!> id=P\\
        &\to& ** a -\!> b -\!> a=P\\
        &\to& ** a -\!> b -\!> a=\&P\\
        &\to& ** a -\!> b -\!> a=\&*P\\
        &\to& ** a -\!> b -\!> a=\&*A\\
        &\to& ** a -\!> b -\!> a=\&*id\\
        &\to& ** a -\!> b -\!> a=\&*d\\
\end{eqnarray*}
\end{enumerate}
