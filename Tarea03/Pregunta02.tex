\textbf{2.} Considera el siguiente fragmento de una gram\'atica que abstrae el 
comportamiento de expresiones del lenguaje \textsf{C}:

\[
E \to *E \;\mid\; \&E \;\mid\; E = E \;\mid\; E\ -\!> E \;\mid\; id
\]

Esta gram\'atica es ambigua pero se puede transformar en una no-ambigua usando
la precedencia de operadores.
En particular, el acceso a campos de una estructura, $E\ -\!> E$, tiene mayor 
precedencia que la derrefenciaci\'on y las expresiones para direcciones; 
adem\'as, estas tres tienen mayor precedencia que las 
asignaciones~\footnote{Puedes revisar la tabla disponible en esta 
\href{https://justdocodings.blogspot.com/2018/06/operator-precedence-and-associativity.html}{p\'agina} 
para consultar la precedencia y asociatividad de los operadores en \textsc{C}.}
\begin{enumerate}
\item Escribe una gram\'atica equivalente tipo \textbf{LL(1)} que incluya la
precedencia descrita, muestra el proceso o describe las técnicas que uses para 
obtener esta nueva gramática.
\item Muestra la tabla de parsing para la gram\'atica del inciso anterior.
\item Procesa la expresi\'on
\verb!* * a -> b -> c = & * d!  usando el algoritmo para \textbf{LL} y 
mostrando los estados del parser. 
\end{enumerate}
