%%% Pregunta 01:
Varios lenguajes de programaci\'on, por ejemplo \textsc{C}, tienen 
definido el enunciado \texttt{switch}:
%\DeclareUnicodeCharacter{2212}{\textendash}
\begin{verbatim}
switch E
   begin
      case V1 : S1
      case V2 : S2
      ...
      case Vn−1 : Sn−1
   default: Sn
end
\end{verbatim}
Describe una forma de traducir este enunciado a c\'odigo de tres direcciones 
(puedes usar saltos y condicionales).
Explica y justifica que el c\'odigo de tres direcciones propuesto respeta el 
mismo comportamiento que el \texttt{switch}.\newline

\textbf{Solución:} Especificamos a $S_i$ y $V_j$, con $i \in [1, \dotsm, n]$
y $j \in [1, \dotsm, n - 1]$, cómo una expresión en código de 3 direcciones
(sino, basta con obtener el código de 3 direcciones para $S_i$ y $V_j$). A
continuación se muestra el código en tres direcciones de la expresión \code{switch}:

\begin{itemize}
\item Para un \code{case} con $V_j$ y $S_i \not= S_{n}$ tenemos que

recorrer($V_j$, $n$):\\
\hspace*{0.2cm} \code{if ($V_j$) goto $S'_i$;} \hspace{6.8cm} con\ $i = j$.\\
\hspace*{0.2cm} \code{else: recorrer($V_j$, $n + 1$);}\\

con $n$ inicialmente en $0$.\\

$S'_i$:\\
\hspace*{0.2cm} \code{$S_i$;}\\
\hspace*{0.2cm} \code{break;}
\item Para el caso \code{default} tenemos que\newline

\code{default:}\\
\hspace*{0.2cm} \code{goto $S'_n$;}\\

\code{$S'_n$:}\\
\hspace*{0.2cm} \code{$S_n$;}\\
\hspace*{0.2cm} \code{break;}
\item Ahora, construyendo el \code{switch} tenemos que\newline

\code{switch ($V_j$):}\\
\hspace*{0.2cm} \code{if ($j < n$) goto recorrer($V_j$, $0$);}\\
\hspace*{0.2cm} \code{else: goto default;}\newline
\end{itemize}

La definición anterior funciona, pues de manera recursiva comprobamos
que siempre que $j < n$ buscamos el ``salto'' a la instrucción $S'_i$
adecuado a $j$ tal que $j = i$. Si $j > n$ entonces se acciona la
intrucción \code{default} que es justamente cómo funciona la instruccón
\code{switch}.
