%%% Pregunta 01:
Varios lenguajes de programaci\'on, por ejemplo \textsc{C}, tienen 
definido el enunciado \texttt{switch}:
%\DeclareUnicodeCharacter{2212}{\textendash}
\begin{verbatim}
switch E
   begin
      case V1 : S1
      case V2 : S2
      ...
      case Vn−1 : Sn−1
   default: Sn
end
\end{verbatim}
Describe una forma de traducir este enunciado a c\'odigo de tres direcciones 
(puedes usar saltos y condicionales).
Explica y justifica que el c\'odigo de tres direcciones propuesto respeta el 
mismo comportamiento que el \texttt{switch}.
