\textbf{5.} (Hasta 1.5pt extra). Explica lo que es una gr\'afica de control de flujo.
?`Qui\'en fue Frances Allen?\\

\textbf{Gráfica de control de flujo}\\

Es una representación visual del flujo de control de un programa de computadora y a su vex proporciona una descripción estructurada de cómo se ejecutan las instrucciones y cómo se toman las decisiones en un programa.\\

En una gráfica de control de flujo, cada instrucción o bloque de código se representa como un nodo, y las conexiones entre los nodos representan las transiciones entre instrucciones, estas transiciones pueden ser causadas por bucles, condicionales, saltos o llamadas a funciones.\\

Los nodos en una gráfica de control de flujo se pueden clasificar en tres tipos principales:

\begin{enumerate}
    \item \textbf{Nodos básicos} : Representan una secuencia lineal de instrucciones sin bifurcaciones o saltos y forman la estructura principal del flujo de control.
    \item \textbf{Nodos de decisión} : Representan puntos en el código donde se toma una decisión basada en una condición y suelen tener ramas o arcos salientes que indican las diferentes opciones que se pueden seguir según el resultado de la condición.
    \item \textbf{Nodos de finalización} : Representan el final de un programa o una función e indican que el flujo de control a terminado.
\end{enumerate}

La gráfica de control de flujo ayuda a los programadores y analistas a comprender la estructura y el comportamiento del programa y también es utilizada en diversas tareas de análisis y optimización de programas, como la detección de bucles, la identificación de código inalcanzable, la localización de cuellos de botella de rendimiento y la verificación de la corrección del programa.\\

Además, la gráfica de control de flujo puede ser utilizada por herramientas de depuración y perfilado para mostrar el recorrido del programa durante la ejecución, ayudando a identificar posibles errores o áreas problemáticas.\\

\textbf{Frances Allen}\\

Frances Allen fue una destacada científica de la computación estadounidense, nació el 4 de agosto de 1932 en Peru, Nueva York, y falleció el 4 de agosto de 2020.\\

Es reconocida por su contribución en el campo de la compilación y optimización de programas además de que en 2006 se convirtió en la primera mujer ganadora del Premio Turing.\\

Fue una pionera en la teoría y la práctica de la compilación que, durante su carrera, trabajó en IBM y fue una de las primeras mujeres en trabajar en programación y desarrollo de software.\\
Su trabajo revolucionó la forma en que se diseñan y optimizan los compiladores, los programas que traducen el código fuente escrito por los programadores a un código ejecutable por las computadoras.\\

Una de las contribuciones más importantes de Frances Allen fue su trabajo en la optimización de programas paralelos y en la identificación de formas eficientes de ejecutar códigos en sistemas de computación paralela.\\
Sus investigaciones y algoritmos siguen influenciando el diseño de compiladores y el desarrollo de técnicas de paralelización de programas hoy en dia.\\